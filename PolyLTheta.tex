\documentclass[12pt]{article}
\usepackage{hyperref}
\usepackage{url}
\usepackage{graphicx}
\usepackage{algorithm}
\usepackage{algpseudocode}
\usepackage{pifont}
\usepackage{amsmath,amssymb}
\usepackage{semantic}
\usepackage{geometry}
\geometry{left=2.5cm,right=2.5cm,top=2.5cm,bottom=2.5cm}

\title{%
  Adding Polymorphic types to L\theta}

\author{Ankit Kumar}

\begin{document}
Updated Syntax :\\

Types\ \ \ $A,B$ ::= Unit $\mid A^{\theta} \rightarrow\ B \mid A+B
\mid A@\theta \mid \alpha \mid \mu \alpha.A \mid \forall \alpha.A$ \\

Terms\ \ \ $a,b$ ::= $x \mid$  rec$\ f\ x.a \mid a\ b \mid$ box$\ a \mid$
unbox$\ x=a$ in $b \mid$ () $\mid$ inl\ $a \mid$ inr$\ a \mid$ case$\ a$\ of\
{inl$\ x \Rightarrow a_1$;inr $\ x \Rightarrow a_2$} $\mid$ roll$\ a$
$\mid$ unroll$\ a \mid \Lambda \alpha.a \mid a[\tau]$  \\

Language Classifiers \ \ \  $\theta$ ::= L $\mid$ P \\

Environments $\Delta\Gamma$ ::= . $\mid $ $\Delta \Gamma$,$x:^{\theta}
A$\\

 \hspace{25 mm}           $ \Delta$   ::=  .$\mid \Delta,\alpha$\\

Values v::= $x \mid $ () $\mid $ inl$\ v \mid $ inr$\ v \mid $ rec$\
f\ x.a \mid $ box$\ v \mid $roll$\ v \mid \Lambda \alpha$.t\\
$$
--------------------------------------------------
$$

TUnivApp
$$
\inference{\Delta\Gamma \vdash^{L} e:\forall \alpha.A \quad \Delta \vdash
B \quad }{\Delta\Gamma \vdash^{L} e[B]:[B/\alpha]A}  
$$

TUnivLam
$$
\inference{\Delta,\alpha,\Gamma \vdash^{\theta} e:B}{\Delta\Gamma
  \vdash^{\theta} \Lambda \alpha.e:\forall \alpha.B}
$$


$$
-------------------------
$$

\section{Good terms}
We define a set $_kGOOD_T$ of terms by induction on type T :
\begin{itemize}
\item t of atomic type T , then t doesn't get stuck for k steps
\item t of arrow type $V \rightarrow W$, $\forall u j,  s.t.\  .\vdash u : V, u
  \rightsquigarrow^j v \wedge irred(v), tv \in\ _{k-j}GOOD_W$
\end{itemize}

\section{Well Behaved candidates}
Now we define Well Behaved candidates of type T, for k steps as set $_kW_T$
such that,\\
(CW 1) If $t \in\  _kW_T$ then t doesn't get stuck before k steps of
computation\\
(CW 2)  If $t \in\  _kW_T$, $t \rightsquigarrow^i t'$ then $t \in\
_{k-i}W_T$.\\
(CW 3) If t is neutral and whenever we convert a redex of $t$, we obtain
some term $t' \in\  _aW_T$ where $a$ is the lowest in all of the
conversions, we have  $t \in\  _{a+1}W_T$.\\

So, we have\\
$$
 _kW_T = \{t | . \vdash t : T,\ t\ satisfies\ CW\ 1,\ CW\ 2\ and\ CW\ 3\}
$$
$$
_kW_X \rightarrow\ _{k-j}W_Y = \{t | . \vdash t : X \rightarrow
Y,\forall j,x, j \leq k, x \in X s.t. x \rightarrow^j v \wedge
irred(v),\ tx \in\ _{k-j}W_Y \}
$$

To Prove: Well behaved terms (which don' get stuck for k steps)
satisfy the given 3 conditions.\\
Proof: By induction on types.\\
1. Atomic Types :
 \begin{itemize}
\item CW 1 : tautology
\item CW 2 : if a computation has run on a term for i steps, only k-i
  steps would remain till it is well. Hence, if $t \in\  _kW_T$, $t \rightsquigarrow^i t'$ then $t \in\
_{k-i}W_T$.
\item CW 3 : If t is neutral and whenever we convert a redex of $t$, we obtain
some term $t' \in\  _aW_T$, where a is lowest in all conversions, then
it is guaranteed that after any conversion, the term will remain well
behaved for atleast a steps. Hence, $t \in\  _{a+1}W_T$.
\end{itemize}

\vspace

2. Arrow types :
\begin{itemize}
\item CW 1: Let $u$ be a variable of type $U$. So, $u$ is neutral and
  normal, hence well behaved for any number of steps. So, $tu \in
  _kGOOD_V$. As $u$ doesn't consume any steps, hence $t$ is well
  behaved for k steps.
\item CW 2 :  Let $u$ be a variable of type $U$. So, $u$ is neutral and
  normal, hence well behaved for any number of steps. Now, $tu
  \rightsquigarrow^i t'u$, so, by I.H. CW 2on V type, $t'u \in
  _{k-i}GOOD_V$. As u didn't consume any step, hence, $t
  \rightsquigarrow^i t'$.
\item CW 3 :  To Prove : $t \in\ _{a+1}W_{U \rightarrow V}$.\\
 Let $u$ be a variable of type $U$. So, $u$ is neutral and
  normal, hence well behaved for any number of steps. So, by I.H. on
  the smaller type $V$, $tu \in\ _{a+1}W_V$ where a is lowest in all
  conversions. As no computation in $u$, hence, $t \in\ _{a+1}W_{U
    \rightarrow V}$.
\end{itemize}


\section{Goodness with parameters}
Let $T[\underbar{X}]$ be a type where $\underbar{X}$ contains atleast
all the free variables in T. Let $\underbar{U}$ represent the sequence
of types, of the same length as $\underbar{X}$. Let $\underbar{G}$ be
the corresponding goodness candidates of types in $\underbar{U}$. Then,

$$
_kGOOD_{X_i}[\underbar{G}/\underbar{X}] =\ _kW_i
$$
$$
_kGOOD_{Y \rightarrow Z}[\underbar{G}/\underbar{X}] =\ _kGOOD_Y[\underbar{G}/\underbar{X}] \rightarrow\ _{k-j}GOOD_Z[\underbar{G}/\underbar{X}]
$$
$$
_kGOOD_{\forall Y.Z}[\underbar{G}/\underbar{X}] = \{t | \forall V, tV
\in\ _kGOOD_Z[\underbar{G}/\underbar{X},\ _kW_V/Y]\}
$$


\textbf{To Prove}: $_kGOOD_{\forall Y.Z}}[\underbar{G}/\underbar{X}] $ is a goodness candidate for type
$T[\underbar{U}/\underbar{X}]$.\\
Proof : If $_kGOOD_{\forall Y.Z}}[\underbar{G}/\underbar{X}] $ is a goodness candidate for type
$T[\underbar{U}/\underbar{X}]$ then it should fulfill CW 1, CW 2 and
CW 3 :\\
(CW 1) Let $t \in _kGOOD_{\forall Y.Z}}[\underbar{G}/\underbar{X}]$.
Then for any type $V$ whose goodness candidate is $_kW_V$,  $tV \in\
_kGOOD_Z[\underbar{G}/\underbar{X},\ _kW_V/Y]$ as defined.\\
So, by Induction Hypothesis on smaller type, $tV$ is good. As there is
no computation within $V$, hence t is good for k steps.\\ 
(CW 2) Let $t \in _kGOOD_{\forall Y.Z}}[\underbar{G}/\underbar{X}]
$ and $t \rightsquigarrow^i t'$. Then for all types $V$ and their
goodness candidate $_kW_V$,  $tV \in\ _kGOOD_Z[\underbar{G}/\underbar{X},\ _kW_V/Y]$ and $tV
\rightsquigarrow^i t'V$.\\
By  Induction Hypothesis on smaller type, $t'V \in\
_{k-i}GOOD_Z[\underbar{G}/\underbar{X},\ _kW_V/Y]$.\\
Hence, $t' \in\ _{k-i}GOOD_{\forall
  Y.Z}}[\underbar{G}/\underbar{X}]$. As $t'$ will be good only for
$k-i$ more steps of computations, so, it satisfies CW 2.\\
(CW 3) Let $t \in _kGOOD_{\forall Y.Z}[\underbar{G}/\underbar{X}]$.
Then for any type $V$ whose goodness candidate is $_kW_V$,  $tV \in\
_kGOOD_Z[\underbar{G}/\underbar{X},\ _kW_V/Y]$ as defined. Now by
I.H. on CW 3, we know that if there are several redexes from tV, then
$tV \in\ _{a+1}GOOD_Z[\underbar{G}/\underbar{X},\ _kW_V/Y]}$ where a is the
minimum of all the other redexes. As no computation occurs in V, we
can argue that $t \in\ _{a+1}GOOD_{\forall Y.Z}[\underbar{G}/\underbar{X},\ _kW_V/Y]}$.

\end{document}

\documentclass[12pt]{article}
\usepackage{hyperref}
\usepackage{url}
\usepackage{graphicx}
\usepackage{algorithm}
\usepackage{algpseudocode}
\usepackage{pifont}
\usepackage{amsmath,amssymb}
\usepackage{mathrsfs }
\usepackage{stmaryrd }
\usepackage{semantic}
\usepackage{geometry}
\usepackage{accents}

\geometry{left=2.5cm,right=2.5cm,top=2.5cm,bottom=2.5cm}

\newcommand{\ub}[1]{\underaccent{\bar}{#1}}
\newcommand{\larr}{\ensuremath{\longrightarrow{}}\xspace}
\newcommand{\arr}{\ensuremath{\,\rightarrow\,}} 

\author{Ankit Kumar}

\begin{document}
Computational Interpretations (CI) in L fragment are essentially
Reducibility Sets, as all terms in CI reduce to a value. So, we can
augment CIs as defined now with a well defined map, mapping types to
their corresponding reducibility sets as shown :
$$
RED_T[\ub{R}/\ub{X}] = _{\beta}\mathscr{C}\llbracket T \rrbracket
$$
where $\beta$ is the mapping $\forall i, X_i \arr
_\beta\mathscr{C}\llbracket X_i \rrbracket$.\\
We will need $\beta$ in the interpretations if there is any type
variables. If not, then we can get rid of $\beta$. e.g. 
$$
X \arr \mathscr{C}\llbracket X \rrbracket
$$
if $X$ has no type vars in it.\\

Now we can define the value interpretation for polymorphic types as
shown:

$$
_\beta\mathscr{V}\llbracket \forall \alpha.A \rrbracket^\theta_k =
\{\wedge \alpha.e| . \vdash \wedge \forall \alpha.e : \forall \alpha.A
\wedge \forall T:type, e[T/\alpha] \in _{\beta[\alpha \arr
  _\beta\mathscr{C}\llbracket T \rrbracket]}\mathscr{C}\llbracket [T/\alpha]A
\rrbracket \}
$$
And so, the proof to Univ-T-app will go according to the definition.\\
Hence the soundness theorem remains unchanged, as defined in the
paper.\\
\\
I am not sure if the interpretation should have $\theta$ in it or just
L.\\
Can we leave the P fragment definition with $j<k$, or would it be better
to use this same definition in P fragment too?

\end{document}
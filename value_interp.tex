\documentclass[12pt]{article}
\usepackage{hyperref}
\usepackage{url}
\usepackage{graphicx}
\usepackage{algorithm}
\usepackage{algpseudocode}
\usepackage{pifont}
\usepackage{amsmath,amssymb}
\usepackage{ mathrsfs }
\usepackage{ stmaryrd }
\usepackage{semantic}
\usepackage{geometry}
\geometry{left=2.5cm,right=2.5cm,top=2.5cm,bottom=2.5cm}

\author{Ankit Kumar}

\begin{document}

Value interpretation for universal type is defined as :\\
$$
\mathscr{V} \llbracket \forall \alpha . A \rrbracket_k^{\theta} =
\{\wedge\alpha.e | . \vdash^{\theta} \wedge \alpha .e : \forall \alpha
  . A\ \wedge \forall T:type, \forall j\leq k \ s.t. \Delta \vdash T, [T/\alpha]e \in
  \mathscr{C}\llbracket [T/\alpha]A \rrbracket^{\theta}_j \}
$$ 

This definition can be used in proof of the Downward Closure Lemma as
shown below :\\
\\
\textbf{To Prove :} if $v \in \mathscr{V}\llbracket \tau \rrbracket^{\theta}_k$
and $j \leq k$ then $v \in \mathscr{V}\llbracket \tau
\rrbracket^{\theta}_j$\\
\textbf{Proof :} By induction on types :\\
\begin{itemize}
\item $\tau$ = Unit\\
Trivial, will satisfy all j as indices are not even part of value
interpretation.
\item $\tau$ = $A^{\theta'} \rightarrow B$ and $\theta$ = L\\
Suppose $i \leq j$ and $v' \in \mathscr{V}\llbracket A^{\theta'} \rrbracket^{L}_i$. \\
Then $i \leq k$ also. Hence, By definition of $\mathscr{V} \llbracket
A^{\theta'}  \rightarrow B \rrbracket^{L}_k$,\\
$[v'/x]a \in \mathscr{C}\llbracket B \rrbracket^{L}_i$. This is true
by definition of $\mathscr{V} \llbracket
A^{\theta'}  \rightarrow B \rrbracket^{L}_j$ also. Hence proved.
\item $\tau = \forall \alpha.A$\\
let $i \leq j$ then $ i \leq k$ too.
if $\wedge\alpha.e \in \mathscr{V} \llbracket \forall \alpha . A
\rrbracket_k^{\theta}$ then  by definition, $\forall T:type \ s.t. \Delta \vdash T, [T/\alpha]e \in
  \mathscr{C}\llbracket [T/\alpha]A \rrbracket^{\theta}_i$ \\
Again, this is the same by definition for $\mathscr{V} \llbracket \forall \alpha . A
\rrbracket_j^{\theta}$ also, hence proved.
\end{itemize}
------------------------------------------------------------------------------------------------\\

Well-formed substitutions are defined as follows :\\
\textbf{Well formed type substitutions :}\\
$$
\mathcal{D}\llbracket . \rrbracket = \{\phi\}
$$
$$
\mathcal{D}\llbracket \Delta,\alpha \rrbracket = \{\rho[\alpha \rightarrow X]\ |\
\rho \in \mathcal{D}\llbracket \Delta \rrbracket \wedge FTV(X) \in dom(\rho) \}
$$
\\
\textbf{Well formed value substitutions :}\\
$$
\mathcal{G}\llbracket . \rrbracket = \{\phi\}
$$
$$
\mathcal{G}\llbracket \Gamma , x:\tau \rrbracket^\rho_k = \{\gamma [x
\rightarrow v]\ |\ \gamma \in \mathcal{G}\llbracket \Gamma
\rrbracket^\rho_k \wedge v \in  \mathscr{V} \llbracket \rho(\tau) \rrbracket_k \}
$$
------------------------------------------------------------------------------------------------
\\

\textbf{T-Univ-Lam rule:}\\
$$
\inference{\Delta,\alpha,\Gamma \vdash^{\theta} e:B}{\Delta\Gamma
  \vdash^{\theta} \Lambda \alpha.e:\forall \alpha.B}
$$
To prove the soundness theorem using T-Univ-Lam case in P fragment:\\
\textbf{Given:} $\Gamma \vdash^{P} \wedge \alpha.e : \forall \alpha.B,\
\Gamma \vDash_k  \gamma,\ \Delta \vDash \rho$\\
\textbf{To Prove:} $\gamma(\rho(\wedge\alpha.e)) \in
\mathscr{C}\llbracket \forall \alpha . B \rrbracket_k^{P}$\\
\textbf{Proof:} From the definition of $\mathscr{C}\llbracket \forall
\alpha . B \rrbracket_k^{P}$, we need to prove
\begin{itemize} 
\item $. \vdash^{P} \gamma(\rho(\wedge\alpha.e)) :  \forall \alpha . B$
  - by appeal to the substitution lemma\\
\item $\gamma(\rho(\wedge\alpha.e)) = \wedge\alpha.\gamma(\rho(e))$ if
  $\alpha$ doesn't occur in $dom(\rho)$.\\
From definition of $\mathscr{C} \llbracket \forall \alpha . A
\rrbracket_k^{P} $, if $\forall T:type,\ \gamma(\rho([T/\alpha]e)) \leadsto^j v$ for all $j < k$,
$\Delta \vdash \rho(\alpha), \Delta \vdash \Gamma$, then it suffices to show that\\
$v \in \mathscr{V}\llbracket \rho([T/ \alpha] B)
\rrbracket_{k-j}^{P}$\\
\\
By Induction Hypothesis,\\
$$
\Delta,\alpha,\Gamma \vdash^{P} e:B
$$\\
$$
\forall T:type,\gamma(\rho([T/\alpha]e)) \in
\mathscr{C}\llbracket\rho( [T/\alpha]B ) \rrbracket_k^{P}
$$\\
Now, we know that $\gamma(\rho([T/\alpha]e)) \leadsto^j v$ and from definition of
$\mathscr{C}\llbracket\rho([T/\alpha] B ) \rrbracket_k^{P}$, we have\\
$v \in \mathscr{V}\llbracket \rho([T/ \alpha] B)
\rrbracket_{k-j}^{P}$.\\
Hence we get that $v \in  \mathscr{V}\llbracket [\rho([T/ \alpha] B)
\rrbracket_{k-j}^{P}$.
\end{itemize}

\\
\\
------------------------------------------------------------------------------------------------\\
\textbf{T-Univ-App rule}\\
$$
\inference{\Delta\Gamma \vdash^{\theta} e:\forall \alpha.A \quad \Delta \vdash
B \quad }{\Delta\Gamma \vdash^{\theta} e[B]:[B/\alpha]A}  
$$
\\
To prove the soundness theorem using T-Univ-App case in L fragment:\\
\textbf{Given:} $\Gamma \vdash^L  e [B] : [B/\alpha]A,\
\Gamma \vDash_k  \gamma,\ \Delta \vDash \rho$\\
\\
\textbf{To Prove:} $\gamma(\rho(e [B])) \in
\mathscr{C}\llbracket [\rho(B)/\alpha]A \rrbracket_k^L$\\
\\
\textbf{Proof:} From the definition of $\mathscr{C}\llbracket
[\rho(B)/\alpha]A \rrbracket_k^L$, we need to prove
\\
\begin{itemize} 
\item $. \vdash^L \gamma(\rho(e [B])) :  [\rho(B)/\alpha]A$
  - By appeal to the substitution lemma\\
\item if $ \gamma(\rho([B/\alpha]e)) \leadsto^{*} v$ ,
$\Delta \vdash B, \Delta \vdash \Gamma$, then it suffices to show that\\
$v \in \mathscr{V}\llbracket [\rho(B)/ \alpha] A
\rrbracket_k^L$\\
\\
\begin{equation} \label{e1}
\gamma(\rho(e [B])) = \gamma(\rho(e)) [\rho(B)]
\end{equation}
if $\alpha$ doesn't occur in $dom(\rho)$.\\
\\
By Induction Hypothesis,\\
$$
\Delta\Gamma \vdash^{L} e:\forall \alpha.A
$$\\
$$
\gamma(\rho(e)) \in
\mathscr{C}\llbracket\forall\alpha.A \rrbracket_k^L
$$\\
By the definition of $\mathscr{C}\llbracket\forall\alpha.A
\rrbracket_k^L$, $\gamma(\rho(e)) \leadsto^{*} v_1 \in \mathscr{V}\llbracket \forall\alpha.A
\rrbracket_k^L$.\\
Hence $\gamma(\rho(e)) [\rho(B)] \leadsto^{*} v_1 [\rho(B)]$ from eq.\ref{e1}.\\
Now as $v_1 \in \mathscr{V}\llbracket \forall\alpha.A \rrbracket_k^L$,it will look like $\wedge\alpha.v'$ for some $v'$ s.t. $\forall
T:Type, v'[T/\alpha] \in \mathscr{C}\llbracket [T/\alpha]A
\rrbracket^L_k$.\\
Initialising $\rho(B)$ in $T$ above, we get $v'[\rho(B)/\alpha] \in \mathscr{C}\llbracket [\rho(B)/\alpha]A
\rrbracket^L_k$.\\
By definition of $\mathscr{C}\llbracket [\rho(B)/\alpha]A
\rrbracket^L_k$, $v'[\rho(B)/\alpha] \leadsto^{*} v'' \in \mathscr{V}\llbracket [\rho(B)/\alpha]A
\rrbracket^L_k$.\\
But $v''$ is actually $v$, hence proved.

\end{itemize}
\end{document}

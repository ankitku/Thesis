\documentclass[12pt]{article}
\usepackage{hyperref}
\usepackage{url}
\usepackage{graphicx}
\usepackage{algorithm}
\usepackage{algpseudocode}
\usepackage{pifont}
\usepackage{amsmath,amssymb}
\usepackage{mathrsfs }
\usepackage{stmaryrd }
\usepackage{semantic}
\usepackage{geometry}
\usepackage{accents}

\geometry{left=2.5cm,right=2.5cm,top=2.5cm,bottom=2.5cm}

\newcommand{\ub}[1]{\underaccent{\bar}{#1}}
\newcommand{\larr}{\ensuremath{\longrightarrow{}}\xspace}
\newcommand{\sgarr}{\ensuremath{\rightsquigarrow{}}\xspace}
\newcommand{\arr}{\ensuremath{\,\rightarrow\,}} 
\newcommand{\llrr}[1]{\llbracket {#1} \rrbracket}

\author{Ankit Kumar}

\begin{document}
All of the names, notations and lemmas used here are taken from ``Proofs and Types'' book and lemmas and theorems will be used here without their proofs.\\

First we prove a result that the computational interpretation of types
in L fragment as defined in Casinghino et al's paper "Step-Indexed Normalization for a
Language with General Recursion" are actually reducibility candidates
of types they are defined for.\\
\\
\textbf{Lemma 1.} $\mathscr{C}\llrr{A}^L_k$ is a reducibility candidate
for type A.\\
\textbf{Proof.} By definition of computational interpretation,
$\mathscr{C}\llrr{A}^L_k$ should satisfy conditions CR 1-3.\\
$$
\mathscr{C}\llrr{A}^L_k = \{a | . \vdash^L a:A \wedge a \sgarr^{*} v \in \mathscr{V}\llrr{A}^L_k\} 
$$
\begin{itemize}
\item{CR-1:} by definition, terms in $\mathscr{C}\llrr{A}^L_k$  reduce
  to a value
\item{CR-2:} If $a \sgarr a'$ then $a' \in \mathscr{C}\llrr{A}^L_k$ as $a' \sgarr^{*} v \in \mathscr{V}\llrr{A}^L_k$
\item{CR-3:} Similarly, if a is neutral, and whenever we convert a
  redex of $a$, we get a term $a' \in \mathscr{C}\llrr{A}^L_k$, then
  it means, $\forall a',\ a \sgarr a' \sgarr^{*} v \in \mathscr{V}\llrr{A}^L_k$, hence
  $a \in \mathscr{C}\llrr{A}^L_k$. \\
As we don't count computation steps in Logical fragment, k is constant
in all cases above.
\end{itemize} 
Parametric Reducibility and its lemmas are defined as in book ``Proofs and
Types'', section 14.\\

We define $\rho$, the mapping of variables to terms :\\
\textbf{Well formed value substitutions :}\\
$$
\mathcal{G}\llbracket . \rrbracket = \{\phi\}
$$
$$
\mathcal{G}\llrr{ \Gamma , x:\tau }_k = \{\rho [x
\rightarrow v]\ |\ \rho \in \mathcal{G}\llrr{ \Gamma
}_k \wedge v \in  \mathscr{V} \llrr{ \tau }_k \}
$$
\\

\textbf{Theorem 1. Soundness : } if $\Gamma \vdash^\theta a:A$ and
$\Gamma \vDash_k \rho$ then\\
\begin{itemize}
\item If a has no type variables or in P fragment, $\rho a \in
  \mathscr{C}\llrr{A}^{\theta}_k$
\item If a has type variables and is in L fragment then, \\
Suppose all the free variables of a are among $x_1,...,x_n$ of types
$U_1,...,U_n$, and all the free type variables of $A,U_1,...,U_n$ are
among $X_1,...,X_m$. If $\mathscr{C}\llrr{V_1}^L_k,...,\mathscr{C}\llrr{V_m}^L_k$ are the reducibility candidates $R_1,...,R_m$ of types $V_1,...,V_m$ and $u_1,...,u_n$ are
terms of types $U_1[\ub{V}/\ub{X}],...,Un[\ub{V}/\ub{X}]$ which are in
$RED_{U_1} [\ub{R}/\ub{X}], . . . , RED_{U_n} [\ub{R}/\ub{X}]$ then  $ a[\ub{V}/\ub{X}][\ub{u}/\ub{x}] \in
RED_T[\ub{R}/\ub{X}]$.
\end{itemize}

\textbf{Proof:} By induction on typing derivations :\\

\begin{itemize}
\item [\textbf{T-UNIV-LAM-L}]$\inference{\Delta,\alpha,\Gamma \vdash^{L} e:B}{\Delta\Gamma
  \vdash^{L} \Lambda \alpha.e:\forall \alpha.B}$\\
By I.H., $\forall \nu:Type,\  e[\ub{V}/\ub{X}][\nu/\alpha][\ub{u}/\ub{x}] \in
RED_B[\ub{R}/\ub{X}][\mathscr{C}\llrr{\nu}^L_k/\alpha]$
From Lemma 14.2.2, $\rho (\wedge \alpha.e[\ub{V}/\ub{X}]) \in
RED_{\forall \alpha.B}[\ub{R}/\ub{X}]$
\item [\textbf{T-UNIV-APP-L}]$ \inference{\Delta\Gamma \vdash^{L} e:\forall \alpha.A \quad \Delta \vdash
B \quad }{\Delta\Gamma \vdash^{L} e[B]:[B/\alpha]A}  $\\
By I.H., $ e[\ub{V}/\ub{X}][\ub{u}/\ub{x}] \in RED_{\forall\alpha.A}[\ub{R}/\ub{X}]$.\\
From Lemma 14.2.3,  $\ \forall B:type,  (e[B][\ub{V}/\ub{X}][\ub{u}/\ub{x}]) \in
RED_{[B/\alpha]A}[\ub{R}/\ub{X}]$\\
\end{itemize}

In the P fragment, the value interpretation for universal type is defined as :\\
$$
\mathscr{V} \llbracket \forall \alpha . A \rrbracket_k^{P} =
\{\wedge\alpha.e | . \vdash^{P} \wedge \alpha .e : \forall \alpha
  . A\ \wedge \forall T:type, \forall j < k \ s.t. \Delta \vdash T, [T/\alpha]e \in
  \mathscr{C}\llrr{[T/\alpha]A}^{P}_j \}
$$ 
\\
\textbf{T-UNIV-LAM-P} : \\
\textbf{Proof:} From the definition of $\mathscr{C}\llbracket \forall
\alpha . B \rrbracket_k^{P}$, we need to prove
\begin{itemize} 
\item $. \vdash^{P} \rho(\wedge\alpha.e) :  \forall \alpha . B$
  - by appeal to the substitution lemma\\
\item $\rho(\wedge\alpha.e) = \wedge\alpha.(\rho(e))$.\\
From definition of $\mathscr{C} \llbracket \forall \alpha . A
\rrbracket_k^{P} $, if $\forall T:type,\ \rho([T/\alpha]e)
\leadsto^j v$ for all $j < k$, then it suffices to show that\\
$v \in \mathscr{V}\llbracket [T/ \alpha] B
\rrbracket_{k-j}^{P}$\\
\\
By Induction Hypothesis,\\
$$
\Delta,\alpha,\Gamma \vdash^{P} e:B
$$\\
$$
\forall T:type,\rho([T/\alpha]e) \in
\mathscr{C}\llbracket [T/\alpha]B  \rrbracket_k^{P}
$$\\
Now, we know that $(\rho([T/\alpha]e)) \leadsto^j v$ and from definition of
$\mathscr{C}\llbracket [T/\alpha] B  \rrbracket_k^{P}$, we have\\
$v \in \mathscr{V}\llbracket ([T/ \alpha] B
\rrbracket_{k-j}^{P}$.\\
Hence we get that $v \in  \mathscr{V}\llbracket [T/ \alpha] B
\rrbracket_{k-j}^{P}$.
\end{document}
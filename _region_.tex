\message{ !name(Polymorphic_Value_Interp.tex)}\documentclass[12pt]{article}
\usepackage{hyperref}
\usepackage{url}
\usepackage{graphicx}
\usepackage{algorithm}
\usepackage{algpseudocode}
\usepackage{pifont}
\usepackage{amsmath,amssymb}
\usepackage{mathrsfs }
\usepackage{stmaryrd }
\usepackage{semantic}
\usepackage{geometry}
\usepackage{accents}

\geometry{left=2.5cm,right=2.5cm,top=2.5cm,bottom=2.5cm}

\newcommand{\ub}[1]{\underaccent{\bar}{#1}}
\newcommand{\larr}{\ensuremath{\longrightarrow{}}\xspace}
\newcommand{\arr}{\ensuremath{\,\rightarrow\,}} 

\author{Ankit Kumar}

\begin{document}

\message{ !name(Polymorphic_Value_Interp.tex) !offset(23) }

And so, the proof to Univ-T-app will go according to the definition.
I am not sure if the interpretation should have $\theta$ in it or just
L.\\
Can we leave the P fragment definition with $j<k$, or would it be better
to use this same definition in P fragment too?

\end{document}
\message{ !name(Polymorphic_Value_Interp.tex) !offset(-34) }

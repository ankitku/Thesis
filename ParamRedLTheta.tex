\documentclass[12pt]{article}
\usepackage{hyperref}
\usepackage{url}
\usepackage{graphicx}
\usepackage{algorithm}
\usepackage{algpseudocode}
\usepackage{pifont}
\usepackage{amsmath,amssymb}
\usepackage{mathrsfs }
\usepackage{stmaryrd }
\usepackage{semantic}
\usepackage{geometry}
\usepackage{accents}

\geometry{left=2.5cm,right=2.5cm,top=2.5cm,bottom=2.5cm}

\newcommand{\ub}[1]{\underaccent{\bar}{#1}}
\newcommand{\larr}{\ensuremath{\longrightarrow{}}\xspace}
\newcommand{\sgarr}{\ensuremath{\rightsquigarrow{}}\xspace}
\newcommand{\arr}{\ensuremath{\,\rightarrow\,}} 
\newcommand{\llrr}[1]{\llbracket {#1} \rrbracket}
\newcommand\tab[1][1cm]{\hspace*{#1}}

\author{Ankit Kumar}

\begin{document}
Let $T[\ub{X}]$ be a type where $\ub{X}$ contains all the free
variables of T. Let $\ub{U}$ be a sequence of types of the same
length; then we can define a type $T[\ub{U}/\ub{X}]$ by simultaneous
substitution of types. Let $PAR\mathscr{C}\llrr{U_i}^\theta_k$ and $PAR\mathscr{V}\llrr{U_i}^\theta_k$ be
the parametrised value and parametrised  computational interpretations of the type
U. Let $\ub{R}$ stand for the parametrised interpretations (P.I.) for
the sequence of types. Then we define the set
$PAR\mathscr{V}\llrr{T}^\theta_k[\ub{R}/\ub{X}]$ and
$PAR\mathscr{C}\llrr{T}^\theta_k[\ub{R}/\ub{X}]$ for terms of type
$T[\ub{U}/\ub{X}]$ as follows :
\begin{itemize}
\item $T = X_i$, then $PAR\mathscr{V}\llrr{T}^\theta_k[\ub{R}/\ub{X}]$= $\mathscr{V}\llrr{U_i}^\theta_k$\\
\tab and $PAR\mathscr{C}\llrr{T}^\theta_k[\ub{R}/\ub{X}]$ =
$\mathscr{C}\llrr{U_i}^\theta_k$
\item $T = Unit$\\
$PAR\mathscr{V}^{\theta}_k[\ub{R}/\ub{X}]  = \{()\}$
\item $T = A+B$\\
$PAR\mathscr{V}\llrr{T}^{\theta}_k[\ub{R}/\ub{X}] = \{inl\ v | v \in
PAR\mathscr{V}\llrr{A}^{\theta}_k[\ub{R}/\ub{X}]\}  \bigcup \{inr\ v | v \in
PAR\mathscr{V}\llrr{B}^{\theta}_k[\ub{R}/\ub{X}]\} $
\item $T = A@\theta'$\\
$PAR\mathscr{V}\llrr{T}^L_k[\ub{R}/\ub{X}] = \{box\ v | v \in PAR\mathscr{V}\llrr{A}^{\theta'}_k[\ub{R}/\ub{X}]\}$

\item $T = A^{\theta'} \arr B $ then,\\
$
PAR\mathscr{V}\llrr{T}^L_k[\ub{R}/\ub{X}] = \{rec\ f\ x.a | . \vdash^L rec\ f\
  x.a : T[\ub{U}/\ub{X}]\ and\ \forall j \leq k,\ if\ v \in
  PAR\mathscr{V}\llrr{A}^{\theta'}_j[\ub{R}/\ub{X}]\ then\ [v/x]a \in  PAR\mathscr{C}\llrr{B}^L_j[\ub{R}/\ub{X}]\}
$\\
$
PAR\mathscr{V}^P_k\llrr{T}[\ub{R}/\ub{X}] = \{rec\ f\ x.a | . \vdash^P rec\ f\
  x.a : T[\ub{U}/\ub{X}]\ and\ \forall j < k,\ if\ v \in
  PAR\mathscr{V}\llrr{A}^{\theta'}_j[\ub{R}/\ub{X}]\ then\ [v/x][rec\
  f\ x.a/f]a \in  PAR\mathscr{C}\llrr{B}^P_j[\ub{R}/\ub{X}]\}
$
\item $T = \mu\alpha.a$\\
$
 PAR\mathscr{V}\llrr{T}^P_k[\ub{R}/\ub{X}] = \{roll\ v | . \vdash^P
 roll\ v: \mu\alpha.a[\ub{U}/\ub{X}]\ and\ \forall j < k,\ v \in
  PAR\mathscr{V}\llrr{[\mu\alpha.a/\alpha]A}^P_j[\ub{R}/\ub{X}]\}
$\\
 $PAR\mathscr{V}\llrr{T}^L_k[\ub{R}/\ub{X}] = \phi$ 
\item $T = \forall\alpha.A$\\
$
PAR\mathscr{V}\llrr{\forall \alpha.a}^{\theta}_k[\ub{R}/\ub{X}] = \{ t |
\forall V:type\ s.t.\ S\ is\ P.I.\ of\ V,\ tV \in PAR\mathscr{C}\llrr{A}^{\theta}_k[\ub{R}/\ub{X},S/Y]\}
$

\end{itemize}

And parametrised computational interpretations are defined as :\\
$$
PAR\mathscr{C}\llrr{T}^P_k[\ub{R}/\ub{X}] = \{a | .\vdash^P
a:T[\ub{U}/\ub{X}]\ and\ \forall j \leq k,\ if\ a\sgarr^j v\ then\ v
\in PAR\mathscr{V}\llrr{T}^P_{k-j}[\ub{R}/\ub{X}]\}
$$
$$
PAR\mathscr{C}\llrr{T}^L_k[\ub{R}/\ub{X}] = \{a | .\vdash^P
a:T[\ub{U}/\ub{X}]\ and\ a \sgarr^{*} v
\in PAR\mathscr{V}\llrr{T}^L_k[\ub{R}/\ub{X}]\}
$$
Observe that Computational Interpretations in the L fragment satisfy
all the CR conditions as given in proofs and types book :\\

\textbf{Lemma 1.} $\mathscr{C}\llrr{A}^L_k$ is a reducibility
candidate for a type A.
for type A.\\
\textbf{Proof.} By definition of computational interpretation,
$\mathscr{C}\llrr{A}^L_k$ should satisfy conditions CR 1-3.\\
$$
\mathscr{C}\llrr{A}^L_k = \{a | . \vdash^L a:A \wedge a \sgarr^{*} v \in \mathscr{V}\llrr{A}^L_k\} 
$$
\begin{itemize}
\item{CR-1:} by definition, terms in $\mathscr{C}\llrr{A}^L_k$  reduce
  to a value
\item{CR-2:} If $a \sgarr a'$ then $a' \in \mathscr{C}\llrr{A}^L_k$ as $a' \sgarr^{*} v \in \mathscr{V}\llrr{A}^L_k$
\item{CR-3:} Similarly, if a is neutral, and whenever we convert a
  redex of $a$, we get a term $a' \in \mathscr{C}\llrr{A}^L_k$, then
  it means, $\forall a',\ a \sgarr a' \sgarr^{*} v \in \mathscr{V}\llrr{A}^L_k$, hence
  $a \in \mathscr{C}\llrr{A}^L_k$. \\
As we don't count computation steps in Logical fragment, k is constant
in all cases above.\\
Hence Parametric Interpretations of types in L fragment are actually
reducibility candidates.\\
We can use Lemmas for Substitution, universal abstraction and
universal application as given in proofs and types book, section 14.2.\\

We now define well formed type and value substitutions :\\
\textbf{Well formed type substitutions :}\\
$$
\mathcal{D}\llbracket . \rrbracket = \{\phi\}
$$
$$
\mathcal{D}\llbracket \Delta,\alpha \rrbracket = \{\rho[\alpha \rightarrow X]\ |\
\rho \in \mathcal{D}\llbracket \Delta \rrbracket \wedge FTV(X) \in dom(\rho) \}
$$
\\
\textbf{Well formed value substitutions :}\\
$$
\mathcal{G}\llbracket . \rrbracket = \{\phi\}
$$
$$
\mathcal{G}\llbracket \Gamma , x:\tau \rrbracket^\rho_k = \{\gamma [x
\rightarrow v]\ |\ \gamma \in \mathcal{G}\llbracket \Gamma
\rrbracket^\rho_k \wedge v \in  \mathscr{V} \llbracket \rho(\tau) \rrbracket_k \}
$$
\\ 
\textbf{New Soundness Theorem : } If $\Delta,\Gamma \vdash^\theta t :
T $ and $\Gamma \vDash_k \rho$, $\Delta \vDash \gamma$ then, $\rho
(\gamma (t)) \in PAR\mathscr{C}\llrr{T}^\theta_k[\ub{R}/\ub{X}]$\\
\textbf{Proof :} By induction on typing derivation. \\
\begin{\itemize}
\item T-UNIV-APP : 
$ \inference{\Delta\Gamma \vdash^{L} e:\forall \alpha.A \quad \Delta \vdash
B \quad }{\Delta\Gamma \vdash^{L} e[B]:[B/\alpha]A}  $\\
 By def.

  $$
PAR\mathscr{C}\llrr{T}^L_k[\ub{R}/\ub{X}] = \{a | . \vdash
  a:T[\ub{V}/\ub{X}]\ and\ a \sgarr^{*} v \in
  PAR\mathscr{V}\llrr{T}^L_k[\ub{R}/\ub{X}] \}
$$
\textbf{(i) : }   $\rho(\gamma (t)) : T[\ub{V}/\ub{X}]$ by appeal to
the substitution lemma.\\
\textbf{(ii) : } for $T = \forall Y.A$ we need to prove that $\rho(\gamma (t)) \sgarr^{*} v \in
  PAR\mathscr{V}\llrr{T}^L_k[\ub{R}/\ub{X}]$\\
Or, from the definition of $PAR\mathscr{V}\llrr{T}^L_k[\ub{R}/\ub{X}]$, $\forall V:type\ s.t.\ S\
is\ C.R.\ of\ V, vV \in PAR\mathscr{V}\llrr{T}^L_k[\ub{R}/\ub{X},S/Y]$\\
\\
From Induction Hypothesis, on T-Univ-App :\\
$\Delta\Gamma \vdash^L t:\forall Y.A$\\
So, $\rho(\gamma(t)) \in PAR\mathscr{C}\llrr{\forall
  Y.A}^L_k[\ub{R}/\ub{X}]$\\
 $ =>\wedge Y.\rho(\gamma(e)) \in PAR\mathscr{V}\llrr{\forall
  Y.A}^L_k[\ub{R}/\ub{X}]$\\
then from definition, $\forall V:type$ and its C.R. $S$,\\
$\rho(\gamma[Y \arr V](e)) \in   PAR\mathscr{V}\llrr{\forall
  Y.A}^L_k[\ub{R}/\ub{X},S/Y]$\\
proved.

\end{document}
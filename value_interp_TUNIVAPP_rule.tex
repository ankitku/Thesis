\documentclass[12pt]{article}
\usepackage{hyperref}
\usepackage{url}
\usepackage{graphicx}
\usepackage{algorithm}
\usepackage{algpseudocode}
\usepackage{pifont}
\usepackage{amsmath,amssymb}
\usepackage{ mathrsfs }
\usepackage{ stmaryrd }
\usepackage{semantic}
\usepackage{geometry}
\geometry{left=2.5cm,right=2.5cm,top=2.5cm,bottom=2.5cm}

\author{Ankit Kumar}

\begin{document}

We have the T-Univ-App rule in L fragment:\\
$$
\inference{\Delta\Gamma \vdash^{L} e:\forall \alpha.A \quad \Delta \vdash
B \quad }{\Delta\Gamma \vdash^{L} e[B]:[B/\alpha]A}  
$$

Where well-formed substitutions are defined as follows :\\
\textbf{Well formed type substitutions :}\\
$$
\mathcal{D}\llbracket . \rrbracket = \{\phi\}
$$
$$
\mathcal{D}\llbracket \Delta,\alpha \rrbracket = \{\rho[\alpha \rightarrow X]\ |\
\rho \in \mathcal{D}\llbracket \Delta \rrbracket \wedge FTV(X) \in dom(\rho) \}
$$
\\
\textbf{Well formed value substitutions :}\\
$$
\mathcal{G}\llbracket . \rrbracket = \{\phi\}
$$
$$
\mathcal{G}\llbracket \Gamma , x:\tau \rrbracket^\rho_k = \{\gamma [x
\rightarrow v]\ |\ \gamma \in \mathcal{G}\llbracket \Gamma
\rrbracket^\rho_k \wedge v \in  \mathscr{V} \llbracket \rho(\tau) \rrbracket_k \}
$$
\\
To prove the soundness theorem using T-Univ-App case:\\
\textbf{Given:} $\Gamma \vdash^L  e [B] : [B/\alpha]A,\
\Gamma \vDash_k  \gamma,\ \Delta \vDash \rho$\\
\\
\textbf{To Prove:} $\gamma(\rho(e [B])) \in
\mathscr{C}\llbracket [\rho(B)/\alpha]A \rrbracket_k^L$\\
\\
\textbf{Proof:} From the definition of $\mathscr{C}\llbracket
[\rho(B)/\alpha]A \rrbracket_k^L$, we need to prove
\\
\begin{itemize} 
\item $. \vdash^L \gamma(\rho(e [B])) :  [\rho(B)/\alpha]A$
  - By appeal to the substitution lemma\\
\item if $ \gamma(\rho([B/\alpha]e)) \leadsto^{*} v$ ,
$\Delta \vdash B, \Delta \vdash \Gamma$, then it suffices to show that\\
$v \in \mathscr{V}\llbracket [\rho(B)/ \alpha] A
\rrbracket_k^L$\\
\\
\begin{equation} \label{e1}
\gamma(\rho(e [B])) = \gamma(\rho(e)) [\rho(B)]
\end{equation}
if $\alpha$ doesn't occur in $dom(\rho)$.\\
\\
By Induction Hypothesis,\\
$$
\Delta\Gamma \vdash^{L} e:\forall \alpha.A
$$\\
$$
\gamma(\rho(e)) \in
\mathscr{C}\llbracket\forall\alpha.A \rrbracket_k^L
$$\\
By the definition of $\mathscr{C}\llbracket\forall\alpha.A
\rrbracket_k^L$, $\gamma(\rho(e)) \leadsto^{*} v_1 \in \mathscr{V}\llbracket \forall\alpha.A
\rrbracket_k^L$.\\
Hence $\gamma(\rho(e)) [\rho(B)] \leadsto^{*} v_1 [\rho(B)]$ from eq.\ref{e1}.\\
Now as $v_1 \in \mathscr{V}\llbracket \forall\alpha.A \rrbracket_k^L$,
it will look like $\wedge\alpha.v'$ for some $v'$ s.t. $\forall
T:Type, v'[T/\alpha] \in \mathscr{C}\llbracket [T/\alpha]A
\rrbracket^L_k$.\\
Initialising $\rho(B)$ in $T$ above, we get $v'[\rho(B)/\alpha] \in \mathscr{C}\llbracket [\rho(B)/\alpha]A
\rrbracket^L_k$.\\
By definition of $\mathscr{C}\llbracket [\rho(B)/\alpha]A
\rrbracket^L_k$, $v'[\rho(B)/\alpha] \leadsto^{*} v'' \in \mathscr{V}\llbracket [\rho(B)/\alpha]A
\rrbracket^L_k$.\\
But $v''$ is actually $v$, hence proved.

\end{itemize}
\end{document}

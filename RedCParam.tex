\documentclass[12pt]{article}
\usepackage{hyperref}
\usepackage{url}
\usepackage{graphicx}
\usepackage{algorithm}
\usepackage{algpseudocode}
\usepackage{pifont}
\usepackage{amsmath,amssymb}
\usepackage{mathrsfs }
\usepackage{stmaryrd }
\usepackage{semantic}
\usepackage{geometry}
\usepackage{accents}

\geometry{left=2.5cm,right=2.5cm,top=2.5cm,bottom=2.5cm}

\newcommand{\ub}[1]{\underaccent{\bar}{#1}}
\newcommand{\larr}{\ensuremath{\longrightarrow{}}\xspace}
\newcommand{\sgarr}{\ensuremath{\rightsquigarrow{}}\xspace}
\newcommand{\arr}{\ensuremath{\,\rightarrow\,}} 
\newcommand{\llrr}[1]{\llbracket {#1} \rrbracket}
\newcommand\tab[1][1cm]{\hspace*{#1}}

\author{Ankit Kumar}

\begin{document}
Let $T[\ub{X}]$ be a type where $\ub{X}$ contains all the free
variables of T. Let $\ub{U}$ be a sequence of types of the same
length; then we can define a type $T[\ub{U}/\ub{X}]$ by simultaneous
substitution of types. Let $PAR\mathscr{C}\llrr{U_i}^\theta_k$ and $PAR\mathscr{V}\llrr{U_i}^\theta_k$ be
the parametrised value and parametrised  computational interpretations of the type
U. Let $\ub{R}$ stand for the candidates of reducibility (C.R.) for
the sequence of types. Then we define the set
$PAR\mathscr{V}\llrr{T}^\theta_k[\ub{R}/\ub{X}]$ and
$PAR\mathscr{C}\llrr{T}^\theta_k[\ub{R}/\ub{X}]$ for terms of type
$T[\ub{U}/\ub{X}]$ as follows :
\begin{itemize}
\item $T = X_i$, then $PAR\mathscr{C}\llrr{T}^\theta_k[\ub{R}/\ub{X}]$ =$R[U_i]^\theta_k$
\item $T = Unit$\\
$PAR\mathscr{V}^{\theta}_k[\ub{R}/\ub{X}]  = \{()\}$
\item $T = A+B$\\
$PAR\mathscr{V}\llrr{T}^{\theta}_k[\ub{R}/\ub{X}] = \{inl\ v | v \in
PAR\mathscr{V}\llrr{A}^{\theta}_k[\ub{R}/\ub{X}]\}  \bigcup \{inr\ v | v \in
PAR\mathscr{V}\llrr{B}^{\theta}_k[\ub{R}/\ub{X}]\} $
\item $T = A@\theta'$\\
$PAR\mathscr{V}\llrr{T}^L_k[\ub{R}/\ub{X}] = \{box\ v | v \in PAR\mathscr{V}\llrr{A}^{\theta'}_k[\ub{R}/\ub{X}]\}$

\item $T = A^{\theta'} \arr B $ then,\\
$
PAR\mathscr{V}\llrr{T}^L_k[\ub{R}/\ub{X}] = \{rec\ f\ x.a | . \vdash^L rec\ f\
  x.a : T[\ub{U}/\ub{X}]\ and\ \forall j \leq k,\ if\ v \in
  PAR\mathscr{V}\llrr{A}^{\theta'}_j[\ub{R}/\ub{X}]\ then\ [v/x]a \in  PAR\mathscr{C}\llrr{B}^L_j[\ub{R}/\ub{X}]\}
$\\
$
PAR\mathscr{V}^P_k\llrr{T}[\ub{R}/\ub{X}] = \{rec\ f\ x.a | . \vdash^P rec\ f\
  x.a : T[\ub{U}/\ub{X}]\ and\ \forall j < k,\ if\ v \in
  PAR\mathscr{V}\llrr{A}^{\theta'}_j[\ub{R}/\ub{X}]\ then\ [v/x][rec\
  f\ x.a/f]a \in  PAR\mathscr{C}\llrr{B}^P_j[\ub{R}/\ub{X}]\}
$
\item $T = \mu\alpha.a$\\
$
 PAR\mathscr{V}\llrr{T}^P_k[\ub{R}/\ub{X}] = \{roll\ v | . \vdash^P
 roll\ v: \mu\alpha.a[\ub{U}/\ub{X}]\ and\ \forall j < k,\ v \in
  PAR\mathscr{V}\llrr{[\mu\alpha.a/\alpha]A}^P_j[\ub{R}/\ub{X}]\}
$\\
 $PAR\mathscr{V}\llrr{T}^L_k[\ub{R}/\ub{X}] = \phi$ 
\item $T = \forall\alpha.A$\\
$
PAR\mathscr{V}\llrr{\forall \alpha.A}^{\theta}_k[\ub{R}/\ub{X}] = \{ t |
\forall V:type\ s.t.\ R[V]^\theta_j\ is\ C.R.\ of\ V,\ tV \in PAR\mathscr{C}\llrr{A}^{\theta}_k[\ub{R}/\ub{X},R[V]^\theta_j/\alpha]\}
$

\end{itemize}

And parametrised computational interpretations are defined as :\\
$$
PAR\mathscr{C}\llrr{T}^P_k[\ub{R}/\ub{X}] = \{a | .\vdash^P
a:T[\ub{U}/\ub{X}]\ and\ \forall j \leq k,\ if\ a\sgarr^j v\ then\ v
\in PAR\mathscr{V}\llrr{T}^P_{k-j}[\ub{R}/\ub{X} ]\}
$$
$$
PAR\mathscr{C}\llrr{T}^L_k[\ub{R}/\ub{X}] = \{a | .\vdash^P
a:T[\ub{U}/\ub{X}]\ and\ a \sgarr^{*} v
\in PAR\mathscr{V}\llrr{T}^L_k[\ub{R}/\ub{X}]\}
$$

\\
\\

Now, for the case
 $T = \forall\alpha.A$\\
$$
PAR\mathscr{V}\llrr{\forall \alpha.A}^{\theta}_k[\ub{R}/\ub{X}] = \{ t
| \forall V:type\ s.t.\ R[V]^\theta_j\ is\ C.R.\ of\ V,\ tV \in PAR\mathscr{C}\llrr{A}^{\theta}_k[\ub{R}/\ub{X},R[V]^\theta_j/\alpha]\}
$$

We need the candidates of reducibility to fulfill conditions :\\
\begin{itemize}
\item $R[A]^{\theta}_k$ needs to be Strongly Normalizing (atleast in
  L, what about P?)
\item If $t \sgarr t'$ and $t' \in R[A]^{\theta}_j,\ then\ t \in
  R[A]^{\theta}_j$.
\end{itemize}


Now we need to define reducibility candidates for each type :
\begin{itemize}
\item T=Unit : $R[Unit]^{\theta}_k = {()}$\\
CR1 and CR2 hold trivially.
\item T=A+B : $R[A+B]^{\theta}_k = \{inl\ a|a \in R[A]^{\theta}_k\} \cup \{inr\ b|b \in R[B]^{\theta}_k\}$\\
CR1 : By I.H. both $ R[A]^{\theta}_k$ and  $ R[B]^{\theta}_k$ are SN.\\
CR2 : from I.H. if $a \in R[A]^{\theta}_k$ then $ a \in
R[A]^{\theta}_j$ by CR2. Similarly $b \in
R[B]^{\theta}_j$ by CR2. Hence $\{inl\ a|a \in R[A]^{\theta}_k\} \cup
\{inr\ b|b \in R[B]^{\theta}_k\} \in R[A+B]^{\theta}_j$.
\item T=$A^{\theta'} \arr B}$ : $R[A^{\theta'} \arr B]^L_k = \{t| t
\sgarr^{*} v \wedge irred(v) \wedge
\forall j \leq k,\ if\ a \in R[A]^{\theta'}_k\ then\ ta\in R[B]^L_j\}$\\
CR1:by def.\\
CR2: let $a \in R[A]^{\theta'}_i$ where $i \leq j \leq k$ then if $t
\in R[A^{\theta'} \arr B]^L_k$ then $ta \in R[B]^L_i$, which is true
also when $t
\in R[A^{\theta'} \arr B]^L_j$
\end{itemize}

\end{document}